%
% 6.006 problem set 1 solutions template
%
\documentclass[12pt,twoside]{article}

\usepackage{amsmath}
\usepackage{color}

\input{macros}

\setlength{\oddsidemargin}{0pt}
\setlength{\evensidemargin}{0pt}
\setlength{\textwidth}{6.5in}
\setlength{\topmargin}{0in}
\setlength{\textheight}{8.5in}

\newcommand{\theproblemsetnum}{1}
\newcommand{\releasedate}{Thursday, February 20}
\newcommand{\partaduedate}{Thursday, February 20}
\newcommand{\tabUnit}{3ex}
\newcommand{\tabT}{\hspace*{\tabUnit}}

\title{6.006 PSET 1}

\begin{document}

\handout{Problem Set \theproblemsetnum}{February 6, 2014}

\textbf{All parts are due {\bf \partaduedate} at {\bf 11:59PM}}.
%
Please download the .zip archive for this problem set, and refer to the
\texttt{README.txt} file for instructions on preparing your solutions.
%
Remember, your goal is to communicate. Full credit will be given only
to a correct solution which is described clearly. Convoluted and
obtuse descriptions might receive low marks, even when they are
correct. Also, aim for concise solutions, as it will save you time
spent on write-ups, and also help you conceptualize the key idea of
the problem.

\setlength{\parindent}{0pt}

\medskip

\hrulefill

\medskip

{\bf Your Name:} Eric Klinkhammer

\medskip

{\bf Collaborators:} Name1, Name2 

\medskip

\hrulefill

\begin{problems}
\section*{Part A}
\problem
\begin{problemparts}
\problempart $(\log n)^{3 \log } <  n^{0.9}\log^3n < 3.3n < n^{3.3}  < (3.3)^n$
\problempart $9^{9n} < 3^{3^n} < 3^{3^{n+1}}< 9^{9^n} < \binom{n}{n/3}$
\problempart $n^9*3^{n/3} < 3^n < n^{n^{1/9}} < n^{n/3} < \left(\frac{n}{3}\right)^{n/3}$
\end{problemparts}
\problem
\begin{problemparts}
\problempart $\Theta(x)$
\problempart $\Theta(x \log x)$
\problempart $\Theta(x \log y)$
\problempart $\Theta( \log x \log y )$
\problempart $\Theta(x\log y + y \log x)$
\end{problemparts}
\problem
\begin{problemparts}
\problempart This algorithm runs in O(n \log n )
\problempart Proof of Correctness\\
Assuming the returned value is both a peak of it's subproblem and the maximum value in the column (of it's subproblem), then it must be a peak 
of the entire problem.  The only way it would not be a peak is if an adjacent value was larger.  Inductively, we know that the value cannot be
located within the subproblem, so we look to the parent problem.  If it were the case that the adjacent element in the larger array were bigger,
then the program would not have looked in the subproblem (having either found the peak or searching on the other side).
\end{problemparts}
\problem
\problem
\problem

\section*{Part B}

\emph{Submit your implemented python script.}

\end{problems}

\end{document}
